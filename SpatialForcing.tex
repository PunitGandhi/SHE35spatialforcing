\documentclass[api,pof,pre,12pt,a4paper]{revtex4-1}     
\usepackage{bm}
\usepackage{natbib}
\usepackage{url}
\usepackage[intlimits]{amsmath}
\usepackage{graphicx}
\usepackage{fancyhdr}
\usepackage{amsfonts}
\usepackage{amssymb}
%\usepackage{pstricks}
%\usepackage{pst-coil}
%\usepackage{pst-plot}
\usepackage{hyperref}
\usepackage{subfig}


\newtheorem{theorem}{Theorem}
\newtheorem{prob}{Problem}
\newenvironment{problem}[1]{\begin{prob} {\rm #1} \end{prob}}

%Nadir's Shortcuts
\newcommand{\beqn}{\begin{equation}}
\newcommand{\eeqn}{\end{equation}}
\newcommand{\beqa}{\begin{eqnarray}}
\newcommand{\eeqa}{\end{eqnarray}}
\newcommand{\beqanonum}{\begin{eqnarray*}}
\newcommand{\eeqanonum}{\end{eqnarray*}}
\newcommand{\beqnonum}{\begin{equation*}}
\newcommand{\eeqnonum}{\end{equation*}}
\newcommand{\jump}{\vspace{0.5cm}}
\newcommand{\bbf}{\begin{bf}}
\newcommand{\ebf}{\end{bf}}
%\newcommand{\eqnref}[1]{(\ref{#1})}
\newcommand{\defn}[1]{\begin{bf}\emph{#1}\end{bf}}
\newcommand{\reals}{\ensuremath{\mathbb{R}}}
\newcommand{\complex}{\ensuremath{\mathbb{C}}}
\newcommand{\integers}{\ensuremath{\mathbb{Z}}}
\newcommand{\half}{\ensuremath{\frac{1}{2}}}
\newcommand{\n}{\nonumber}
\renewcommand{\d}{\mathrm{d}}
\newcommand{\del}{\partial}
\newcommand{\dd}{\ensuremath{\, \mathrm{d}}}
\newcommand{\nint}[4]{\int_{#3}^{#4} {#1}\, \mathrm{d}{#2}}
\newcommand{\der}[2]{\frac{\d {#1}}{\d {#2}}}
\newcommand{\parder}[2]{\frac{\del {#1}}{\del {#2}}}
\newcommand{\funder}[2]{\frac{\delta {#1}}{\delta {#2}}}
\newcommand{\Lag}{\mathcal{L}}

\oddsidemargin  0.0in
\evensidemargin 0.0in
\textwidth      6.5in
\headheight     15pt
\topmargin      0.0in
\textheight=8.0in
%\setlength{\parindent}{0in}

\lhead{Spatially periodic forcing in SH35}
\chead{}
\rhead{Gandhi et al.}
%\lfoot{}
%\cfoot{}
%\rfoot{}


%Figures
\newcommand{\FIGcowboy}{
\begin{figure}[t]\center
\includegraphics[width=60mm]{CowboyHatPotential.png}
\caption{\label{fig:CowboyHat} A plot of the cowboy hat potential with $l=0.005$, $\mu=1.0$, and $\gamma=0.5$.}
\end{figure}
}
\newcommand{\FIGforce}{
\begin{figure}[t]\center
\includegraphics[width=60mm]{CowboyField.png}
\caption{\label{fig:CowboyForce} A plot of the force corresponding to the term of the potential that breaks radial symmetry.}
\end{figure}
}

\newcommand{\FIGpotcont}{
\begin{figure}[t]\center
\includegraphics[width=60mm]{CowboyPotCont.png}
\caption{\label{fig:CowboyCont} A plot of the contours of the potential of $\bar{h}$ of Eq. \ref{eq:CowboyHat}. The darker color corresponds to lower energy.}
\end{figure}
}
\newcommand{\FIGpotcontL}{
\begin{figure}[t]\center
\includegraphics[width=60mm]{PotContL0.png}
\caption{\label{fig:PotContL0} A plot of the contours of the potential of $\bar{h}$ of Eq. \ref{eq:CowboyHat} in the case of zero angular momentum. The darker color corresponds to lower energy.}
\end{figure}
}

\newcommand{\FIGsnakeknot}{
\begin{figure}[htp]
  \centering
\caption{  \label{fig:snakeknot} bifurcation diagrams with the L2 norm vs forcing strength (r) for different strengths of periodic forcing perturbation.}

  \subfloat[$\delta=0.00$]{\label{fig:snakeknot:1}\includegraphics[width=60mm]{sb00.png}}
  \subfloat[$\delta=0.02$]{\label{fig:snakeknot:2}\includegraphics[width=60mm]{sb02.png}}
  \\
  \subfloat[$\delta=0.04$]{\label{fig:snakeknot:3}\includegraphics[width=60mm]{sb04.png}}
  \subfloat[$\delta=0.06$]{\label{fig:snakeknot:4}\includegraphics[width=60mm]{sb06.png}}
\end{figure}
}

\newcommand{\FIGsnakecollapse}{
\begin{figure}[htp]
  \centering
\caption{  \label{fig:snakecollapse} bifurcation diagrams with the L2 norm vs forcing strength (r) for different strengths of periodic forcing perturbation.}

  \subfloat[$\delta=0.12$]{\label{fig:snakeknot:1}\includegraphics[width=60mm]{sb12.png}}
  \subfloat[$\delta=0.14$]{\label{fig:snakeknot:2}\includegraphics[width=60mm]{sb14.png}}
  \subfloat[$\delta=0.16$]{\label{fig:snakeknot:3}\includegraphics[width=60mm]{sb16.png}}
 \end{figure}
}


\pagestyle{fancy}
\begin{document}
\preprint{APS/123-QED}

\title{Spatially periodic forcing in the Swift Hohenberg equation}

\author{Punit Gandhi}
 \email{punit_gandhi@berkeley.edu}
\author{Hsien-Ching Kao}
\author{C\'edric Beaume}
\author{Edgar Knobloch}
 \email{knobloch@berkeley.edu}
\affiliation{Department of Physics, University of California, Berkeley CA 94720, USA}

\begin{abstract}
We study the effect of a spatially periodic perturbation to the forcing term in the cubic-quintic Swift-Hohenberg equation (SH35).  We focus our analysis near the bifurcation point of the patterned state from the homogeneous state and develop an amplitude equation to describe the averaged dynamics at leading order.  We also use numerical continuation to look at how the behavior of localized states is effected by the perturbation.
\end{abstract}

\maketitle

\section{Introduction}
The Swift-Hohenberg equation serves as a model for pattern formation in a broad range of physical systems.  We will be interested in the case of a cubic-quintic nonlinearity term, which allows for the existence of localized states.  This equation, which takes the form  
\begin{equation}
u_t= r u-\left(1+\partial_{x}^2\right)^2u+bu^3-u^5\label{eq:SH35},
\end{equation}
describes the dynamics of a real field $u$ over one spatial dimension in time.  We have rescaled the equation, which will be referred to as SH35, so that the critical wavenumber that defines the natural wavelength of the patterned state is unity. The strength of the linear forcing term $r$ and the strength of the cubic nonlinearity $b$ are left as parameters of the system.  {\it (Some more details about work that has already been done on this equation)} 


Realistic systems will not always have a perfectly homogeneous forcing in space, and inhomogeneities may even lead to a way to manipulate dynamics of localized states. As a simple extension to Eq. \ref{eq:SH35}, we would now like to consider the case when the forcing has a periodic perturbation,
\begin{equation}
u_t= r(1+\delta \cos{\kappa x}) u -\left(1+\partial_{x}^2\right)^2u+bu^3-u^5\label{eq:SH35spf}.
\end{equation}

We can make some analytic progress on studying Eq.~\ref{eq:SH35spf} using techniques very similar to those used on the original equation (Eq.~\ref{eq:SH35}). We will look near the $r=0$ bifurcation point on a long timescale and a long spatial scale to get an amplitude equation at leading order.  In the special case that the periodicity of the forcing is half that of the natural wavelength, we obtain a parametrically forced Ginsburg-Landau equation for the amplitude.  


\section{Preliminary Analysis}
The addition of the periodic forcing perturbation does not destroy the variational structure of Eq. \ref{eq:SH35}.  Indeed, Eq. \ref{eq:SH35spf} can still be written in terms of a Lyapunov functional as 
\beqn
u_t=-\frac{\delta F}{\delta u},
\eeqn
where the Lyapunov functional F has been slightly modified
\beqn
F=\int -\frac{1}{2}r(1+\delta \cos\kappa x) u^2+\frac{1}{2}\left[(1+\partial_x^2) u\right]^2 -\frac{1}{4}bu^4+\frac{1}{6}u^6\; dx.
\eeqn
Therefore, we know that the system will still relax towards a steady state in time.  This motivates the study of time-independent solutions that will be pursued in following sections.

The trivial solution $u=0$ also still exists for the perturbed system (Eq. \ref{eq:SH35spf}). There can no longer be nonzero constant solutions as there were for certain parameter regimes of the original SH35.  To see this, we need only assume a constant solution for $u$ and we will run into  a contradiction when the constant is not zero (assume $\delta$ is nonzero).  


\section{Scaling}
We would like to look at small amplitude solutions in the neighborhood of the $r=0$ bifurcation where the periodic state branches off of the homogeneous state in the space of steady-state solutions.  We will take a multi-scale approach, defining a slow timescale $T=\epsilon^2t$, and long spatial scale $X=\epsilon x$ so that the derivatives become $\partial_t \rightarrow \partial_t+\epsilon^2\partial_T$ and $\partial_x \rightarrow \partial_x+\epsilon\partial_X$.  We will assume that the system will not change on the fast timescale, so we can neglect the $\partial_t$ term. With some trial and error, it can be seen that the appropriate scaling of forcing strength to probe the dynamics we are interested in will be $r=\epsilon^2 \mu$.     Additionally, we will not make any assumptions about $\delta$ yet, but will eventually want to consider the case that $\epsilon <<\delta <<1$.

With this scaling, Eq. \ref{eq:SH35spf} becomes 
\begin{equation}
\epsilon^2 u_T = \epsilon^2 \mu(1+\delta \cos{\kappa x}) u -\left(1+(\partial_{x}+\epsilon \partial_X)^2\right)^2u+bu^3-u^5
\label{eq:SH35spfeps}.
\end{equation}
Note we will also assume that the spatial periodicity of the forcing happens at a lengthscale comparable that of the characteristic wavelength of the patterned state.  In particular, we will focus on the case that $\kappa=2$, as this will lead a parametrically forced Ginzburg-Landau equation.

\section{Expansion}
We will write out the solution as an asymptotic expansion in the small parameter $\epsilon$:
\begin{equation}
u=\sum_{j=0}^{\infty} \epsilon^j u_j
\end{equation}
Since we want to look for small amplitude solutions, we will assume that $u_0=0$ and that the leading order solution comes in only at $\mathcal{O}(\epsilon)$. Furthermore, we can see by the structure of the nonlinear terms that we will not need the even powers of the expansion.  The equation that $u_2$ will need to satisfy, for example, will be identical to the one that $u_1$ will need to satisfy so that we could effectively redefine it by $u_1\rightarrow u_1+\epsilon u_2$.  We will finally note that the leading order effects of the spatially modulated forcing will by captured by making the expansion out to order $\mathcal{O}(\epsilon^3)$.   
\begin{equation}
u=\epsilon u_1 + \epsilon^3 u_3  + \mathcal{O}(\epsilon^5)
\label{eq:AsympExpU}
\end{equation}

We can now plug in the asymptotic series form of the solution Eq. \ref{eq:AsympExpU}) into the scaled version of the SHE35 (Eq. \ref{eq:SH35spfeps}) and match terms in orders of $\epsilon$.  The leading order terms will be $\mathcal{O}(\epsilon)$:
\begin{equation}
u_1+2\partial_x^2 u_1 +\partial_x^4 u_1 = 0.
\end{equation}
This equation has a solution of the form
\begin{equation}
u_1(x,X,T)=A(X,T) e^{i x}+ \bar{A}(X,T) e^{-ix},
\label{eq:u1sol}
\end{equation}
where $A(X,T)$ is a still undetermined complex amplitude that depends only on the slow timescale $T$ and the long lengthscale $X$, and $\bar{A}(X,T)$ is its complex conjugate. Collecting terms at the next order in $\epsilon$ will provide a solvability condition that we will use to determine $A$.

We can now go on to look at the next terms in the expansion, which come in at $\mathcal{O}(\epsilon^3)$:
\begin{eqnarray}
u_3+2\partial_x^2 u_3 &+& \partial_x^4 u_3 =  \nonumber \\
& & -\partial_T u_1-2\partial_X^2 u_1 - 6\partial_x^2 \partial_X^2 u_1 +\mu u_1 +\mu \delta \cos(\kappa x)  u_1 +b u_1^3
\label{eq:AsympExpEps3}
\end{eqnarray}
Plugging in the form of the solution for $u_1$ from Eq. \ref{eq:u1sol} gives the following expression for the RHS of the equation above.
\begin{eqnarray}
\text{RHS}&=&  \nonumber \\ 
& &\left(-A_T  +4 A_{XX}+\mu A + 3b |A|^2 A\right)e^{ix} + \left(-\bar{A}_T +4 \bar{A}_{XX}+\mu \bar{A} + 3b |A|^2 \bar{A}\right)e^{-ix} \nonumber \\
& &+\frac{\mu\delta}{2} A\left(e^{i(1+\kappa)x}+e^{i(1-\kappa)x} \right) + \frac{\mu\delta}{2} \bar{A}\left( e^{-i(1+\kappa)x}+e^{-i(1-\kappa)x} \right) \nonumber \\
& & +bA^3e^{3ix}+b\bar{A}^3e^{-3ix}
\label{eq:RHSeps3}
\end{eqnarray}
We must now make a choice about $\kappa$ in order to proceed, and  assuming it to be a (non-negative) rational number will allow for progress via Fourier projections.  If we assume that $\kappa=m/n$ for relatively prime integers $m$ and $n$, we can use the projection operator $\tfrac{1}{2\pi n}\int_{-n\pi}^{n\pi} \text{d} x \; e^{-i x}$ to find a solvability condition for the amplitude $A$.  The condition that this project vanishes results in the following equation for determining $A$:
\begin{equation}
A_T= 4 A_{XX}+\mu\left(1+\delta \delta_d(\kappa)\right) A + 3b |A|^2 A+\tfrac{\mu\delta}{2} \delta_d(\kappa-2)\bar{A}
\label{eq:Aeq}
\end{equation}
where $\delta_d$ is the Dirac delta function.  The choice of $\kappa$ determines if the effects of the spatially periodic forcing perturbation can be seen at this order.  In particular, there is no effect from the periodic forcing perturbation at this order when $\kappa$ is a positive integer other than 2.  The choice of $\kappa=2$ on the other hand, modifies the complex amplitude equation at this order so that it includes a parametric forcing term (proportional to $\bar{A}$).  We could also consider the case of rational fraction values of $\kappa$, which would generically also have a parametric forcing term added.  This would correspond to forcing perturbations with wavelength longer than the characteristic wavelength of the pattern. This seems useful, as it could allow one control a small-scale pattern by generating a larger scale one.  

\section{The 2:1 resonance in the forcing}
We will now focus on the case that the period of the forcing perturbation is half that of the characteristic wavelength of the patterned state, namely that $\kappa=2$.  Plugging this into Eq. \ref{eq:AsympExpEps3} leads to the following equation for the complex amplitude from the solvability condition that resonant terms vanish:
\begin{equation}
-A_T  +4 A_{XX}+\mu A + 3b |A|^2 A+\frac{\mu\delta}{2}\bar{A}=0
\label{eq:LGEpf1}
\end{equation}

With appropriate redefinitions of variables, we can put this equation into the following form
\begin{equation}
A_T  = \mu A + A_{XX} - |A|^2 A+\gamma\bar{A}
\label{eq:LGEpf}
\end{equation}
where we have assumed that $b<0$. {\it Should I look at the $b>0$ case instead?}

\subsection{Polar Coordinates}
Equation \ref{eq:LGEpf}  can now be put into a form that resembles a particle orbiting in central potential by making the substitution $A=r e^{i\theta}$, where $r$ and $\theta$ are real functions of $X$ and $T$.  The resulting pair of real pde's is:
\begin{subequations}
\begin{align}
\dot{r}&=\mu r - r^3 +r''-r\theta'^2+\gamma r \cos2\theta 
\label{eq:PolarGLEpfr} \\
r^2\dot{\theta}&=2 r r' \theta'+r^2\theta''-\gamma r^2 \sin 2\theta
\label{eq:PolarGLEpfth}
\end{align}
\end{subequations}
where the dot and prime represent derivatives with respect to the time $T$ and the position $X$, respectively.

\subsection{Constant amplitude solutions}
The original SH35 has periodic solutions that are time-independent and have a constant amplitude in space.  We can try to look for similar solutions to this case by assuming $r=r_0$ is a nonzero constant. Substituting this into Eq. \ref{eq:PolarGLEpfr} leads to
\beqn
\theta'^2=\mu-r_0^2+\gamma \cos2\theta
\label{eq:r0Polar1}
\eeqn
which,upon taking a derivate, gives
\beqn
\theta''=-\gamma \sin2\theta
\eeqn
On the other hand, Eq. \ref{eq:PolarGLEpfth} reduces to 
\beqn
\theta''=\gamma \sin2\theta
\eeqn
after making the constant amplitude assumption.  The only way for both of these equations to be true, assuming a nonzero $\gamma$, is for $\sin2\theta=0$. This implies that $\theta=n\pi/2$ must be constant, where $n$ is some integer.  Plugging value this of $\theta$ back into Eq. \ref{eq:r0Polar1} gives the corresponding possible values of the amplitude, $r_0=\sqrt{\mu +(-1)^n\gamma}$. Note that we only consider positive values of $r$ since a negative value corresponds to a $\pi$ shift in $\theta$. These solutions will exist in the parameter regime where $\mu\pm\gamma\ge 0$.  These constant solutions are straightforward to find from the original amplitude equation (Eq. \ref{eq:LGEpf}).  It is not difficult to see that the solutions $A=\pm i \sqrt{\mu-\gamma}$ are stable in space when they exist (i.e. $\mu \geq \gamma$), and the solutions $A=\pm \sqrt{\mu+\gamma}$ are unstable in space when they exist (i.e. $\mu \geq-\gamma$).


The solutions found here are constant in space and time for the amplitude, which correspond to periodic solutions to the perturbed SH35 (Eq. \ref{eq:SH35spf}). Explicitly, the leading order solutions are 
\beqn
u_1=\pm \frac{\sqrt{\mu-\gamma}}{2} \sin x,\: \pm \frac{\sqrt{\mu+\gamma}}{2} \cos x
\eeqn
The $\sin$ solutions are stable in space and the $\cos$ solutions are unstable in space, but we need to go back and look at their stability in time.



\subsection{Time-independent solutions and spatial dynamics}
If we look for steady state solutions in time, we can reduce this problem to an ode with the dependent variable as space instead of time.  We can define a spatial {\it energy} $h$ and {\it angular momentum} $l$ by:
\begin{eqnarray}
l &=& r^2\theta' \\
h &=& \frac{1}{2} r'^2 +\frac{l^2}{2r^2} +\frac{1}{2}\mu r^2 -\frac{1}{4} r^4
\end{eqnarray}
In the limit that $\gamma$ vanishes, these new variables are actually constants of motion of the amplitude Eq. \ref{eq:LGEpf}.  The problem described by Eq. \eqref{eq:PolarGLEpfr} and \eqref{eq:PolarGLEpfth} can be mapped onto the problem of a particle traveling in a central potential equivalent to a particle in a central potential if we consider dynamics in space instead of time.  The spatial energy $h$ has a kinetic term, a effective potential term from the spatial angular momentum $l$, and a central potential well term.  Equations \eqref{eq:PolarGLEpfr} and \eqref{eq:PolarGLEpfth} can now be rewritten in terms of the spatial energy and angular momentum as:
\begin{subequations}
\begin{align}
l' &= \gamma r^2 \sin 2\theta 
\label{eq:CentPotL} \\
h'&= \gamma l \sin 2\theta-\gamma r r' \cos 2\theta
\label{eq:CentPotH}
\end{align}
\end{subequations}

Assuming $\epsilon <<\gamma <<1$ allows us to treat $l$ and $h$ as slowly varying quantities.  We could, in principle, apply the method of averaging to this equation to get spatially averaged equations that depend only on $l$ and $h$.  We would assume that these two quantities are constant and integrate over one period to average out the small variations of $r$ and $\theta$.  Unfortunately, this involves integrals that I'm not exactly sure how to do.

We can use the interpretation of a particle in a central potential to help us find solutions to our original problem that are localized in space and steady-state in time.  These type of solutions correspond to  orbits of the particle that approach $r=0$ as $x\rightarrow \pm \infty$.  We could look for these kinds of solutions to the unperturbed problem (when $\gamma=0$) and then try to understand what the perturbation does to these solutions. 

We can further simplify the energy equation by noting that the RHS is the spatial derivative of the quantity $-\half \gamma r^2 \cos 2 \theta$. By redefining the energy as
\beqn
\bar{h}=\frac{1}{2} r'^2 +\frac{l^2}{2r^2} +\frac{1}{2}\mu r^2 -\frac{1}{4} r^4+\half \gamma r^2 \cos 2 \theta,
\label{eq:CowboyHat}
\eeqn
so that the potential is no longer radially symmetric.  A plot of this potential, which will be referred to as the {\it cowboy hat potential}, is shown in Fig. \ref{fig:CowboyHat}.
\FIGcowboy
The force corresponding the additional energy term is $\mathbf{F}=-\gamma r (\cos2\theta \;\hat{r} - \sin2\theta \; \hat{\theta})$ is plotted in Fig. \ref{fig:CowboyForce}.    
\FIGforce
We can aslo plot contours of the modified potential energy (Eq. \ref{eq:CowboyHat}) to help us visualize what  the orbits in the complex $A$ space might look like. (Fig. \ref{fig:CowboyCont}).
\FIGpotcont

\subsection{Localized states}
We can search for states that are localized in space by searching for the equivalent orbits in the particle in potential picture described above.   In this interpretation, the localized states correspond to orbits that approach $r=0$ as $x\rightarrow \pm \infty$.  Such an orbit is only possible when $l=0$ so that the value of the potential at $r=0$ does not blow up.  In this case, $\theta$ must be constant and we are looking for orbits that just move radially.  The a contour plot of the potential of $\bar{h}$  when $l=0$,  $\mu=-0.2$, and $\gamma=0.5$ is shown in Fig. \ref{fig:PotContL0}.  
\FIGpotcontL
It is clear that the only possible way for the orbit we are searching for is when $\theta=\pm \pi/2$, and in this case the particle would have $\bar{h}=0$.  With the parameter values $\mu=-0.2$ and $\gamma=0.5$, the maximum amplitude for the localized state can be found numerically to be $r\approx 1.2$.

\section{Numerical Work}
Numerical Continuation can be used to follow solutions that are steady in time over a range of parameters.  We will focus on such bifurcation diagrams that vary the forcing parameter $r$, and will use the $L2$ norm as a way to characterize the solutions so that they can be plotted.  We can make a series of these diagrams as for different values of strength of the spatially periodic perturbation $\delta$.


There are two interesting features of the series of plots described above, which we would like to understand.  Firstly,  Fig. \ref{fig:snakeknot} shows that two distinct sections of the snaking region develop as the perturbation is increased from $\delta=0.00$ to $\delta=0.06$.  The upper region of the snaking branch is shifted to the left relative to the lower region, and this shift grows as the perturbation increases.  In fact, the shift appears to be linear with respect to the forcing strength.  Hsien-Ching has developed an analytic estimation of this kind of effect in a similar situation, and thinks the same technique can be used here.  His prediction was that the shift is linearly proportional to the forcing perturbation strength to leading order which is exactly what we see.  Furthermore, we see the upper section takes over more and more of the lower section as the perturbation increases.  We also notice that there is initially a knot separating these two regions, but eventually gets pulled out.  Finally, it is interesting to note that the lower section corresponds to localized patterns of the characteristic wavelength will the upper section has localized structures with a longer wavelength. One other peculiar feature is seen in Fig. \ref{fig:snakeknot:2} when $\delta=0.02$.  The top section of the snaking branch, which consists of solutions of longer wavelength, eventually reconnects with the periodic branch of characteristic wavelength.
\FIGsnakeknot

The second feature of interest can be seen at slightly higher values of the forcing $\delta\approx 0.15$. As the two sections continue to separate from each other,  the top section starts to unfold into two separate snaking branches and then eventually collapses so that there is only a lower section of snaking.  This happens between the ranges of $\delta=0.12$ and $\delta=.16$.
\FIGsnakecollapse


\section{Future Work}
I still need to find references for work on the parametrically forced Ginzburg-Landau equation.

I should do the calculation for SHE23.  It might also be nice to take SHE35 calculation to higher order to see the effect of the nonlinearity.   In this case, $\kappa=3$ will probably also have an interesting effect.

The next thing to do might be to look for localized solutions by looking for parameter regimes where it is possible for the orbit to approach 0 as $x \rightarrow \pm \infty$.  I've started to do this, but notices some peculiarities.  I need to look more closely at the $l=0$ case for a wider range of parameters.

Another interesting thing to do might be to set up a time-stepping code for the SH35 with the periodic forcing and then see how some of the steady-state solutions found numerical continuation evolve in time when the parameters are shifted slightly, the solutions are modified slightly, or if some noise is added to them. We have a time-stepping code now, and just need to play around with it a little. This code will also be able to import solutions from auto, which will be a very nice feature.

One other thing I'd be interested in, is to see what happens if the forcing is slightly off from the 2:1 resonance. Hsien Ching has given me some suggestions about how this might be possible to do, and it seems like it will lead to an addition forcing term that depends on the slowly varying spatial scale.

It might be interesting to consider a system where the spatially periodic forcing actually comes from another pattern forming system.  This could involve having two pattern forming fields that each satisfy the Swift-Hohenberg equation, and are coupled  by  forcing term linearly proportional to both fields.  If one of the systems were in a patterned state, it could provide the kind of perturbation that we are looking at here.
    



\bibliography{SpatialForcingBib}
\end{document}




